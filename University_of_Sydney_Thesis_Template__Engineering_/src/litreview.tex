\chapter{Literature review\label{cha:litreivew}}

Text.

\section{Synapse Detection \& Segmentation}

Synapse detection and segmentation in volume electron microscopy constitute a foundational stage for reliable connectome reconstruction. In this section, the focus is explicitly on electron‐microscopy data, with an emphasis on \emph{serial-section electron microscopy} (ssEM) and \emph{focused ion beam scanning electron microscopy} (FIB--SEM). We formalize a three-tier objective: (i) \emph{synaptic site or cleft detection}, that is, localizing presynaptic and postsynaptic specializations or the interposed cleft; (ii) \emph{instance-level or boundary-aware segmentation} of synaptic structures, including active zones, postsynaptic densities, and vesicle clusters, at voxel or supervoxel resolution; and (iii) \emph{partner assignment} as an optional stage that links each detected site to its presynaptic and postsynaptic neurites, thereby instantiating edges in the neural graph. Practical deployment is complicated by extreme class imbalance and the small, sparse, and morphologically diverse nature of synapses; anisotropy in ssEM and contrast drift across large volumes; weak or partial supervision (points, line segments, or sparse cleft masks) that limits dense annotation; and pronounced domain shift across species, brain regions, staining protocols, and acquisition sites. In addition, segmentation and assignment errors propagate topologically into neuron-level reconstructions and downstream graph analyses, while constraints on compute and human proofreading cost impose requirements on throughput, calibration, and robustness. These considerations motivate methods that integrate multi-scale context, enforce structural consistency, and generalize across datasets.

\noindent\textbf{Roadmap.} The remainder of this section follows a common template (definition $\rightarrow$ methods $\rightarrow$ evaluation $\rightarrow$ failure modes $\rightarrow$ emerging trends) and is organized into six parallel subsections: \emph{Problem scope and data regime}, which clarifies inputs, outputs, supervision regimes, and noise characteristics; \emph{Classical pipelines}, which cover feature-engineered detectors, Markov and conditional random fields, and supervoxel or graph-based post-processing; \emph{Deep learning paradigms}, which span two- and three-dimensional U-Net families, transformer backbones, two-stage candidate-to-refinement pipelines, and end-to-end instance models with uncertainty estimation; \emph{Structural priors and constraints}, which include biological context, connectivity- and topology-preserving objectives, and multi-task consistency; \emph{Benchmarks, metrics and failure modes}, which unify detection, segmentation, topology, and graph-level metrics while cataloguing typical errors; and \emph{Trends and open issues}, which highlight integrated detect–segment–assign objectives, cross-domain self-supervision, and interactive correction workflows.

\subsection{Problem scope and data regime}

\noindent\textbf{Task definition}\;
This subsection formalizes synapse analysis in volume electron microscopy with emphasis on serial section electron microscopy (ssEM) and focused ion beam scanning electron microscopy (FIB--SEM). We adopt a three-tier objective that maps voxels to synaptic structures and then to edges in a connectome:
(i) \emph{synaptic site or cleft detection}, i.e., localizing presynaptic specializations, postsynaptic densities, or the interposed cleft;
(ii) \emph{instance-level or boundary-aware segmentation} of synaptic components (active zones, postsynaptic densities, vesicle clusters);
(iii) \emph{partner assignment} that links each detected site to its presynaptic and postsynaptic neurites to instantiate synaptic edges.
Unless stated otherwise, polyadic synapses (one presynaptic site with many postsynaptic partners) are allowed; self-synapses and inhibitory/excitatory types are not distinguished at the label level in this section.\par

\medskip
\noindent\textbf{Definition and notation}\;
Let the EM volume be \(x \in \mathbb{R}^{H \times W \times D}\) with voxel spacings \(\Delta=(\Delta_x,\Delta_y,\Delta_z)\) in nanometres.
A voxel index \(\mathbf r=(i,j,k)\) corresponds to the physical coordinate \(\tilde{\mathbf r}=(i\Delta_x,\,j\Delta_y,\,k\Delta_z)\).
We denote the set of site candidates by \(\mathcal P\), the synapse mask by \(S \subset \{1,\dots,H\}\times\{1,\dots,W\}\times\{1,\dots,D\}\), and presynaptic/postsynaptic subsets by \(\mathcal P_{\mathrm{pre}}\) and \(\mathcal P_{\mathrm{post}}\).
The partner set is \(\mathcal E \subseteq \mathcal P_{\mathrm{pre}}\times \mathcal P_{\mathrm{post}}\).
All geometric quantities (distances, radii, areas, volumes) are stated in \emph{physical units}.
We view the three tasks as
\[
\text{Detection: } x \mapsto \mathcal P,\qquad
\text{Segmentation: } x \mapsto S,\qquad
\text{Assignment: } (\mathcal P,S) \mapsto \mathcal E.
\]\par

\medskip
\noindent\textbf{Inputs}\;
The input consists of volumetric EM acquired under two prevalent regimes.
ssEM typically exhibits pronounced anisotropy along the sectioning axis (\(\Delta_z \gg \Delta_x,\Delta_y\)), which complicates 3D context aggregation and boundary continuity.
FIB--SEM approaches near-isotropic sampling, improving volumetric continuity while sometimes offering lower in-plane contrast for fine ultrastructure relative to ssEM.
In selected settings, light microscopy and correlative LM--EM provide auxiliary cues for registration, cross-modal consistency checks, or weak supervision (\cite{Simhal2017PCBiol}); the primary target here remains EM volumes.\par

\medskip
\noindent\textbf{Preprocessing and tiling (for reproducibility)}\;
To standardize inputs and enable scalable inference, we assume: global or per-tile intensity normalization; optional denoising; slice-to-slice alignment for ssEM; artifact masking (folds, cracks, missing sections); and resampling if needed for network stride compatibility.
Block-wise inference uses fixed window size, stride, and overlap with consensus reconciliation on overlaps; these values are reported in physical units.\par

\medskip
\noindent\textbf{Outputs and matching policy}\;
We distinguish three outputs that may be produced jointly or sequentially.
\emph{Detection} yields point-like sites or small spherical/box proposals.
\emph{Segmentation} returns voxel masks for synaptic compartments, with optional membrane-contact constraints.
\emph{Partner assignment} produces a set of edges \(\mathcal E\) with one-to-one, one-to-many (polyadic), or many-to-many cardinalities, as predefined in protocol.
Let \( d_{\mathrm{phys}}(p,g)=\|\tilde{\mathbf r}(p)-\tilde{\mathbf r}(g)\|_2 \).
Detection matches are declared if \(d_{\mathrm{phys}}(p,g)\le r\) with a default tolerance radius \(r\) specified in nanometres and accompanied by a sensitivity sweep.
Partner legality requires (i) nonzero membrane contact between presynaptic and postsynaptic neurites; and (ii) thresholds on minimum contact area and maximum centroid distance, both in physical units.\par

\medskip
\noindent\textbf{Supervision regimes}\;
Annotation is frequently incomplete and heterogeneous: points, short line segments tracing clefts or active zones, or sparse masks for selected subvolumes.
Many datasets provide weakly supervised local blocks rather than globally dense labels, and cross-laboratory or cross-species scenarios often provide no target-domain labels (\cite{Park2022Cerebellar}).
Learning must therefore contend with severe class imbalance and a preponderance of very small objects (\cite{Svara2022Zebrafish}).\par

\medskip
\noindent\textbf{Noise characteristics and acquisition drift}\;
Real-world volumes show contrast fluctuations across tiles or sessions, section misalignment and residual warping (ssEM), staining artefacts, stripe noise or deposition, focus drift, and resampling-induced deformations from alignment or block merging.
These factors distort local texture statistics and long-range context, undermining both candidate generation and boundary closure; robust training typically includes augmentation and quality-gating for these artefacts.\par

\medskip
\noindent\textbf{Interfaces to downstream tasks}\;
Synapse analysis is coupled to neuron segmentation and to graph construction.
Synapse mis-localization can induce false splits or merges in neurite agglomeration, while neuron-mask errors can misassign partners; both effects propagate topologically and alter circuit motifs and path lengths.
Edge existence and weight estimation depend on partner assignment and on calibrated detection scores; miscalibration at the detector level induces systematic errors in graph-level metrics (degree distributions, modularity, motif counts).
Evaluations should therefore consider voxel/instance accuracy alongside graph fidelity and human proofreading cost.\par

\medskip

\subsection{Classical pipelines}

\noindent\textbf{Scope and three-stage layout}\;
Classical (pre-deep) synapse pipelines typically follow a three-stage design:
(i) \emph{preprocessing \& candidate detection} (templates/filters/morphology, e.g., DoG, normalized cross-correlation, anisotropic kernels);
(ii) \emph{segmentation \& component extraction} (thresholding \(+\) connected components, watershed, active contours, superpixel/supervoxel graphs);
(iii) \emph{structured inference} (MRF/CRF, \(s\)-\(t\) graph cuts, shortest paths, or integer programming).
This subsection focuses on these canonical designs and their reproducible details; end-to-end deep models are discussed elsewhere.

\medskip
\noindent\textbf{Preprocessing and normalization}\;
Common steps include slice-to-slice registration (ssEM), denoising (nonlocal means, BM3D, anisotropic diffusion), intensity standardization (histogram matching or per-tile normalization), and artefact masking (folds, cracks, missing sections).
Block-wise inference specifies window size, stride, and overlap; cross-block reconciliation uses max/avg fusion or non-maximum suppression (NMS).
All sizes are reported in physical units.

\medskip
\noindent\textbf{Hand-crafted feature families}\;
Given voxel intensities \(I(\mathbf r)\) and scale-space \(I_\sigma = G_\sigma * I\),
\[
g_\sigma(\mathbf r)=\lVert \nabla I_\sigma(\mathbf r)\rVert_2,\qquad
\Delta I_\sigma,\qquad
\mathbf H_\sigma=\nabla^2 I_\sigma.
\]
Features are grouped as \emph{intensity/contrast}, \emph{gradient/edge} \((g_\sigma,\,\Delta I_\sigma)\), \emph{texture} (Gabor, LM filter bank, LBP), \emph{morphology} (opening/closing, distance transform, skeletons), and \emph{structure-tensor/Hessian} cues for sheet-/tube-like responses.
A typical sheet-response uses the Hessian eigenvalues \(\lambda_1\le\lambda_2\le\lambda_3\):
\[
R_{\text{sheet}}
=\exp\!\Big(-\tfrac{\lambda_1^2+\lambda_2^2}{2\alpha^2}\Big)\cdot
\Big(1-\exp\!\big(-\tfrac{\lambda_3^2}{2\beta^2}\big)\Big).
\]

\medskip
\noindent\textbf{Candidate detection: DoG and NCC}\;
Difference-of-Gaussians (DoG) at scales \(\sigma_1,\sigma_2\):
\[
D(\mathbf r;\sigma_1,\sigma_2)=(G_{\sigma_1}-G_{\sigma_2})*I,\qquad
\mathbf r^\star=\arg\max_{\mathbf r} D(\mathbf r;\cdot)\ \ \text{s.t.}\ \ D>\tau.
\]
Normalized cross-correlation (NCC) with template \(T\) over window \(\Omega\):
\[
\mathrm{NCC}(\mathbf r)=
\frac{\sum_{\mathbf u\in\Omega}\!\big(I(\mathbf r+\mathbf u)-\bar I_\Omega\big)\big(T(\mathbf u)-\bar T\big)}
{\sqrt{\sum_{\mathbf u\in\Omega}\!\big(I(\mathbf r+\mathbf u)-\bar I_\Omega\big)^2}\,
 \sqrt{\sum_{\mathbf u\in\Omega}\!\big(T(\mathbf u)-\bar T\big)^2}}.
\]
Peaks are selected across multiple scales with anisotropic kernels (heavier \(z\)-weights for ssEM) and pruned by NMS.

\medskip
\noindent\textbf{Segmentation models: from thresholding to structured inference}\;
Starting from thresholding \(+\) connected components or watershed seeds, classical methods lift decisions into graphical models.
For binary labels \(y_i\in\{0,1\}\) (non-synapse/synapse), a standard CRF/MRF energy is
\[
E(\mathbf y)=\sum_i \psi_i(y_i\,|\,x)+\sum_{(i,j)\in\mathcal E}\psi_{ij}(y_i,y_j),
\]
with unary \(\psi_i=-\log P(y_i\,|\,\phi(I))\) from hand-crafted features \(\phi\),
and pairwise Potts \(\psi_{ij}=\lambda\,\mathbf 1[y_i\neq y_j]\cdot w_{ij}\) where \(w_{ij}\) depends on gradient, boundary, or membrane likelihood.
If \(\psi_{ij}\) is a metric Potts, the optimum is found via \(s\)-\(t\) graph cuts; otherwise, \(\alpha\)-expansion or loopy BP gives approximate inference.
Shortest-path stitching and region-merging provide alternatives when thin interfaces fragment.

\medskip
\noindent\textbf{Supervoxels and region adjacency graphs (RAG)}\;
Watershed or SLIC yields supervoxels as nodes of a graph \(\mathcal G=(\mathcal V,\mathcal E)\).
Edge weights \(w_{uv}\) combine boundary strength, average gradient, texture disparity, or membrane probability.
Bottom-up merging applies a criterion
\[
\Delta \mathcal J=\Delta\mathrm{Data}-\gamma\,\Delta\mathrm{Boundary},
\quad \text{merge if }\Delta \mathcal J>\tau_{\text{merge}},
\]
with sparse \(z\)-linking for anisotropic stacks.

\medskip
\noindent\textbf{Active contours / level sets (optional)}\;
Energy consists of a data term and length/curvature regularization; evolution proceeds until a stopping rule (e.g., small update norm) is met.
They are useful when shapes are smooth but edges are weak, and can be seeded from candidate sites.

\medskip
\noindent\textbf{Topology/bio priors as post-processing}\;
Rules include membrane-contact constraints (reject non-membrane candidates), minimum contact area and maximum inter-centroid distance, hole filling and thin-layer preservation (morphological repair), and 3D connectivity consistency across slices.

\medskip
\noindent\textbf{Strengths and limitations}\;
\begin{itemize}\setlength\itemsep{0.25em}
  \item \emph{Strengths:} interpretability of features/energies; low label demand; fast training and modest memory; straightforward rule injection (smoothness, contiguity, simple geometric priors).
  \item \emph{Limitations:} weak cross-domain generalization; limited capacity for context and topology (polyadic relations, membrane–cleft–vesicle dependencies); difficulty with multi-scale and very small targets; cascading errors between candidate and consolidation stages; calibration often suboptimal for proofreading cost.
\end{itemize}

\medskip
\noindent\textbf{Typical instantiations and scenarios}\;
\begin{itemize}\setlength\itemsep{0.25em}
  \item \emph{Texture \(+\) MRF for candidate generation.} Multiscale textures or steerable filters scored by RF/SVM, followed by MRF/CRF smoothing, suit small anisotropic ssEM volumes with scarce labels and moderate cleft contrast.
  \item \emph{Supervoxel posteriors \(+\) graph consolidation.} Watershed/SLIC on membrane/cleft likelihoods, then graph cuts or shortest paths to fuse fragments, supports batch processing of small subvolumes and manual curation.
  \item \emph{Low-annotation regimes.} Random-forest scoring of putative clefts or T-bar–like structures \(+\) CRF remains a pragmatic baseline when only points/lines/sparse masks are available or for rapid prototyping.
\end{itemize}

\medskip
\noindent\textbf{Parameter selection and ablations}\;
Report thresholds, scale sets \(\{\sigma\}\), template sizes, NMS radius, Potts weight \(\lambda\), edge-weight construction, RAG merge threshold \(\tau_{\text{merge}}\), shortest-path costs, and contact-area/distance thresholds (all in physical units).
Use grid or Bayesian search; provide sensitivity curves and cross-validation design.

\medskip
\noindent\textbf{Evaluation protocol (aligned with later sections)}\;
Detection uses a physical matching radius \(r\) (default \(r=\)\textit{XX}\,nm; include \(r\)-sweep).
Segmentation reports Dice, VOI, Rand \(F\); boundary reports Boundary-F1 with tolerance \(\delta=\)\textit{YY}\,nm.
Provide PR/AP for candidates, instance-level \(F_1\), and, if applicable, graph consistency.
Attach 95\% confidence intervals (bootstrap over instances/tiles) and hardware/run-time/memory.

\medskip
\noindent\textbf{Runtime and resources}\;
State throughput (voxels/s), peak memory, single-/multi-core speedups, and parallelism (tiling/pipelining); specify hardware and implementation language.

\medskip
\noindent\textbf{Failure modes and boundary conditions}\;
Common issues include staining drift, strong artefacts, extreme anisotropy, low contrast, and confounds from vesicles or myelin, leading to false positives/negatives.
Mitigations include adaptive thresholds, region rejection, and local re-estimation around hard cases.

\medskip
\noindent\textbf{Interface to deep methods}\;
Classical modules provide priors, pseudo-labels, and post-processors for deep systems (e.g., CRF/graph-cut refinement), and serve as weak supervision or fallbacks in low-label regimes.

\medskip
\noindent\textbf{Algorithm C1: reference classical pipeline}\;
\begin{enumerate}
  \item \textbf{Preprocess}: registration \(\rightarrow\) denoise \(\rightarrow\) intensity standardization \(\rightarrow\) artefact masking.
  \item \textbf{Candidate detection}: DoG/NCC/morphology (multi-scale, anisotropic) \(\rightarrow\) NMS for points/patches.
  \item \textbf{Feature extraction}: compute \(\phi(I)\) (intensity/texture/gradient/Hessian/morphology).
  \item \textbf{Unary scoring}: train/apply RF/SVM/AdaBoost to obtain \(P(y_i\,|\,\phi)\).
  \item \textbf{Structured segmentation}: CRF/graph-cuts/watershed\(+\)RAG on candidate neighborhoods or full volume.
  \item \textbf{Topology-aware postprocess}: membrane-contact checks, contact-area/distance thresholds, hole repair, 3D connectivity.
  \item \textbf{Threshold \& calibration}: PR/ROC sweeps; choose operating point; optional temperature/Platt scaling.
  \item \textbf{Reporting}: metrics ( \(r,\delta\), Dice/VOI/RandF/AP ), run-time/memory, sensitivity and ablations.
\end{enumerate}

\medskip
\begin{table}[t]
  \centering
  \scriptsize
  \setlength{\tabcolsep}{2pt}%
  \renewcommand{\arraystretch}{1.05}%
  \begin{tabular}{@{} p{0.15\linewidth} p{0.21\linewidth} p{0.17\linewidth} p{0.20\linewidth} p{0.22\linewidth} @{}}
    \hline
    \textbf{Family} & \textbf{Operators / models} & \textbf{Priors / constraints} & \textbf{Advantages} & \textbf{Limitations / failures} \\
    \hline
    Templates / filtering &
    DoG, LoG, NCC, Gabor, morphology &
    Multi-scale, anisotropic kernels &
    Simple; fast; light memory &
    Sensitive to contrast/protocol shifts; threshold dependent \\
    Shallow classifiers &
    RF / SVM / AdaBoost on \(\phi(I)\) (intensity/gradient/texture/Hessian) &
    Class weights; imbalance handling &
    Interpretable; controllable; small label demand &
    Hand-crafted features needed; weak domain transfer \\
    Structured models &
    CRF/MRF (Potts), graph cuts, shortest paths, integer programming &
    Smoothness; boundary; membrane-contact priors &
    Good boundary consistency; topology control &
    Many hyperparameters; approximate inference; tuning cost \\
    Region methods / RAG &
    Watershed\(+\)RAG; hierarchical merging; MST-style agglomeration &
    Intra-region coherence; boundary penalty; sparse \(z\)-links &
    Enables instance merging; scalable on blocks &
    Seed/damping sensitive; over-merge risk; seed bias \\
    Active contours / level sets &
    Chan--Vese; geodesic AC; curvature regularization &
    Length/curvature priors; shape bias &
    Robust to weak edges with proper seeds &
    Leakage under strong noise; parameter sensitive \\
    \hline
  \end{tabular}
  \caption{Classical method families in synapse analysis. Keep physical units for any thresholds; insert concrete implementations and citations in the camera-ready version.}
  \label{tab:classical-compare}
\end{table}

\subsection{Deep learning paradigms}

\noindent\textbf{Scope and task decomposition}\;
This subsection covers deep-learning paradigms for three coupled goals:
\emph{detection} (point/heatmap/center-offset representations),
\emph{segmentation} (semantic/instance/boundary-aware masks),
and \emph{partner assignment} (pre--post linkage and graph edges).
Compared with classical three-stage pipelines, deep systems may be \emph{end-to-end} (joint heads for all goals) or \emph{two-stage} (candidate proposal \(\rightarrow\) refinement/post-processing), while retaining clear interfaces to classical modules when desired.\par

\medskip
\noindent\textbf{Network archetypes and anisotropy-aware design}\;
We group backbones into four families and make the anisotropy choices explicit for ssEM versus near-isotropic FIB--SEM:
\begin{itemize}
  \item \emph{2D/2.5D U-Net family}: in-plane convolutions with slice stacks (2.5D) or occasional axial \(1\times 1\times k\) kernels; memory-efficient, high throughput; limited axial context.
  \item \emph{3D U-Net / hybrid 2.5D}: anisotropic kernels \(k\times k\times 1\) in early stages and delayed \(z\)-downsampling; stronger 3D context; higher memory and more complex tiling.
  \item \emph{CNN+Transformer hybrids} (e.g., windowed in-plane attention with axial gating): large effective receptive fields and cross-domain robustness; training stability and memory must be managed.
  \item \emph{Multi-head multi-task decoders} (FPN/DeepLab-like): shared encoder with parallel heads for detection/segmentation/boundary/pairing; enables end-to-end coupling but requires careful loss balancing.
\end{itemize}
For strongly anisotropic ssEM, we use mixed kernel shapes (e.g., \(k\times k\times 1\) interleaved with \(1\times 1\times k\)), delay \(z\)-strides until deeper levels, or apply axial attention/gating to capture long-range slice structure.\par

\medskip
\noindent\textbf{Heads for the three goals}\;
\emph{Detection (anchor-free preferred)}: a center heatmap \(\hat{Y}^{\mathrm{det}}\) and offsets \(\hat{O}\) (optionally size/radius), peak finding \(\rightarrow\) NMS with a \emph{physical} suppression radius; in crowded tiny-object regimes, soft-NMS or weighted fusion reduces mutual suppression.
\emph{Segmentation}: semantic logits \(\hat{S}\) (Dice/CE/Tversky objectives) and optional boundary head \(\hat{B}\); instance masks via center-offset clustering, energy/level-set proxy, or proposal-based heads.
\emph{Partner assignment}: embed pre/post candidates into \(\hat{Z}\) (metric learning) or predict pairwise affinities \(\hat{A}\); training uses local \emph{soft-assignment} within a radius \(r\), while inference may apply Hungarian or min-cost flow, with membrane-contact and distance/area rules.\par

\medskip
\noindent\textbf{Losses and balancing (formalized)}\;
\emph{Detection \& segmentation}:
\[
L_{\text{det}}
=\mathrm{FL}\big(\hat{Y}^{\text{det}},Y^{\text{det}}\big)
+\beta\,\lVert \hat{O}-O\rVert_{1,\mathcal{N}},\qquad
L_{\text{seg}}
=(1-\mathrm{Dice}(\hat{S},S))+\alpha\,\mathrm{CE}(\hat{S},S)+\eta\,L_{\text{bd}},
\]
where \(L_{\text{bd}}\) regularizes boundaries (e.g., distance-transform bands of width \(\delta\) or a level-set proxy).
\emph{Pairing (two options)}:
\[
\text{(embeddings)}\quad
L_{\text{pair}}
=\sum_{(i,j)\in\mathcal{N}_r}\ell_{\text{pos}}(\hat{Z}_i,\hat{Z}_j)
+\sum_{(i,k)\notin\mathcal{N}_r}\ell_{\text{neg}}(\hat{Z}_i,\hat{Z}_k),
\]
\[
\text{(affinities)}\quad
L_{\text{pair}}
=-\!\!\sum_{(i,j)\in\mathcal{N}_r}\!\!\log\sigma(\hat{A}_{ij})
-\!\!\sum_{(i,k)\notin\mathcal{N}_r}\!\!\log\!\big(1-\sigma(\hat{A}_{ik})\big),
\]
where \(\mathcal{N}_r\) collects candidate pairs within physical radius \(r\).
\emph{Topological/biological priors (optional)}:
\[
L_{\text{topo}}
=\lambda_{\mathrm{conn}}\mathcal{L}_{\mathrm{conn}}(\hat{S})
+\lambda_{\mathrm{hole}}\mathcal{L}_{\mathrm{hole}}(\hat{S})
+\lambda_{\mathrm{mc}}\mathcal{L}_{\mathrm{membrane\_contact}}(\hat{S},M),
\]
with \(M\) a membrane probability/mask, implemented via soft-skeleton or distance transforms.
\emph{Calibration}:
\[
\hat{p}^{(T)}=\mathrm{softmax}(z/T),\qquad
\mathrm{ECE}=\sum_{b=1}^{B}\frac{n_b}{n}\,\bigl|\mathrm{acc}(b)-\mathrm{conf}(b)\bigr|.
\]
\emph{Total loss}:
\[
L_{\text{total}}
=\lambda_{\text{det}}L_{\text{det}}
+\lambda_{\text{seg}}L_{\text{seg}}
+\lambda_{\text{pair}}L_{\text{pair}}
+\lambda_{\text{topo}}L_{\text{topo}}
+\lambda_{\text{cal}}\mathrm{ECE},
\]
with \(\lambda_{\cdot}\) chosen by grid search, uncertainty weighting, or GradNorm, and sensitivity reported.\par

\medskip
\noindent\textbf{Anisotropy and 2.5D specifics (quantified)}\;
When \(\Delta_z\gg \Delta_x,\Delta_y\): use kernel schedules such as early \(k\times k\times 1\) (no \(z\)-downsampling) then \(k\times k\times 2\) at depth \(N\); adopt slice stacks of depth \(d\) in 2.5D encoders; and include axial attention/gating with window \(k\) to keep memory bounded. Downstream NMS and pairing radii are specified in \emph{physical} units to avoid voxel-size confounds.\par

\medskip
\noindent\textbf{Augmentation, semi-supervision, and domain adaptation}\;
Intensity (brightness/contrast/gamma), histogram matching, \emph{missing-slice simulation}, in-plane rotations/flips, elastic warps, light scaling, cutout/cutmix, and EM-like noise injection.
For semi-supervision, adopt teacher--student consistency (weak/strong views) with confidence filtering and simple morphology checks; for domain adaptation, combine histogram/style transfer with feature alignment (e.g., MMD/adversarial) as lightweight options.\par

\medskip
\noindent\textbf{Calibration and uncertainty for QA}\;
Apply temperature scaling or Platt scaling on validation, report ECE/ACE, and fix operating thresholds accordingly. Optional deep ensembles or MC dropout provide voxel/edge-level uncertainty used to filter pseudo-labels and to prioritize human proofreading.\par

\medskip
\noindent\textbf{Training protocol (reproducible essentials)}\;
State patch size in physical units (with halo), batch measured in voxels, optimizer (SGD/AdamW), LR schedule (cosine/multistep), mixed precision, gradient accumulation, EMA, and hard-example mining.
Specify positive/negative sampling ratios and class weights to counter extreme imbalance.\par

\medskip
\noindent\textbf{Inference, tiling, and block fusion}\;
Use sliding windows with halo; optional test-time augmentation (in-plane flips/rotations).
Fuse overlaps by averaging or Gaussian weighting; for detection, apply NMS in \emph{physical} radius and deduplicate across blocks; refine masks with optional CRF and reconcile instances across tiles.\par

\medskip
\noindent\textbf{Evaluation protocol (aligned with the chapter)}\;
Detection: PR/AP and \(F_1@r\) under a physical matching radius \(r\) with sensitivity sweeps.
Segmentation: Dice/mIoU, Boundary-F1 with tolerance \(\delta\), plus VOI/Rand \(F\) for instance topology.
Pairing/graph: edge PR/ROC/AUC and calibration curves; report 95\% CIs via bootstrap over instances/tiles.\par

\medskip
\noindent\textbf{Runtime and resources}\;
Report throughput (Mvox/s), peak memory, parameter count/FLOPs, and latency; list hardware and library versions.
Document tile/halo and cross-block reconciliation, as they affect both accuracy and speed.\par

\medskip
\noindent\textbf{Failure modes and mitigations}\;
Low contrast, strong artefacts/ghosting, and polyadic crowding yield misses or merges; mitigations include boundary-head reinforcement, adaptive thresholds, local re-estimation around hard regions, and topology-aware regularizers.\par

\medskip
\noindent\textbf{Interfaces to the rest of the thesis}\;
Losses, matching rules, and calibration follow the definitions in \emph{Problem scope and data regime} and \emph{Benchmarks, metrics \& failure modes} to prevent protocol drift.\par

\medskip
\noindent\textbf{Box D1: encoder--decoder with anisotropic kernels (example)}\;
Backbone with 4--5 levels; alternate in-plane \(k\times k\times 1\) and axial \(1\times 1\times k\); delay the first \(z\)-stride to level \(N\).
Heads: detection \(\hat{Y}^{\text{det}},\hat{O}\), segmentation \(\hat{S}\) and boundary \(\hat{B}\), pairing \(\hat{Z}\) or \(\hat{A}\).\par

\medskip
\noindent\textbf{Algorithm D1: training}\;
\begin{enumerate}
  \item \textbf{Sample}: crop patches (physical units) with halo; stratify hard regions; fix pos/neg ratios and class weights.
  \item \textbf{Augment}: intensity/geometric/noise; if semi-supervised, use weak/strong consistency and filter pseudo-labels.
  \item \textbf{Forward}: backbone \(+\) heads \(\rightarrow \hat{Y}^{\text{det}},\hat{O},\hat{S},\hat{B},\hat{Z}/\hat{A}\).
  \item \textbf{Loss}: compute \(L_{\text{det}},L_{\text{seg}},L_{\text{pair}},L_{\text{topo}}\) and ECE; update with chosen weighting.
  \item \textbf{Optimize}: AdamW/SGD with cosine or multistep LR; mixed precision, accumulation, optional EMA.
  \item \textbf{Select thresholds/temperature}: sweep on validation; fix operating points and calibration temperature.
\end{enumerate}

\noindent\textbf{Algorithm D2: inference}\;
\begin{enumerate}
  \item \textbf{Tiling}: sliding windows with halo; optional in-plane TTA; fuse outputs (mean/Gaussian).
  \item \textbf{Detection}: peak/connected components \(\rightarrow\) NMS (physical radius) and cross-block deduplication.
  \item \textbf{Segmentation}: threshold/optional CRF refinement; instance formation via offsets/energy/region merging.
  \item \textbf{Pairing}: build local cost within radius \(r\) from \(\hat{Z}/\hat{A}\), solve with Hungarian/min-cost flow; filter by contact area/distance.
  \item \textbf{Calibration}: apply temperature \(T\), export PR/ROC and calibration curves with confidence intervals.
\end{enumerate}

\medskip
% ---- Compact tables that fit typical thesis layouts (no extra packages) ----
\begin{table}[t]
  \centering
  \scriptsize
  \setlength{\tabcolsep}{2pt}\renewcommand{\arraystretch}{1.05}
  \begin{tabular}{@{} p{0.19\linewidth} p{0.43\linewidth} p{0.16\linewidth} p{0.20\linewidth} @{}}
    \hline
    \textbf{Archetype} & \textbf{Core design} & \textbf{Pros} & \textbf{Caveats} \\
    \hline
    2D/2.5D U-Net & In-plane conv; slice stacks; axial \(1\times 1\times k\) & Memory/throughput & Limited \(z\)-context \\
    3D U-Net / hybrid & Anisotropic kernels; delayed \(z\)-stride & Strong 3D cues & High memory; complex tiling \\
    CNN+Transformer & Windowed in-plane attn \(+\) axial gating & Large RF; robust & Stability/memory tuning \\
    Multi-head (FPN) & Shared encoder; heads for det/seg/boundary/pairing & End-to-end & Loss balancing \\
    \hline
  \end{tabular}
  \caption{Deep architecture patterns with anisotropy-aware choices.}
  \label{tab:dl-arch-compare}
\end{table}

\begin{table}[t]
  \centering
  \scriptsize
  \setlength{\tabcolsep}{2pt}\renewcommand{\arraystretch}{1.05}
  \begin{tabular}{@{} p{0.23\linewidth} p{0.52\linewidth} p{0.23\linewidth} @{}}
    \hline
    \textbf{Target} & \textbf{Typical loss} & \textbf{Notes} \\
    \hline
    Detection & Focal/weighted BCE \(+\) L1/L2 offsets & Extreme imbalance; tiny objects \\
    Semantic seg. & Dice \(+\) CE/Tversky & Foreground sparsity; calibration \\
    Boundary seg. & Level-set proxy / dist.-transform / Boundary-F1 surrogate & Tolerance band \(\delta\) \\
    Pairing & Contrastive/triplet/InfoNCE or BCE on \(\hat{A}\) & Local soft-assign; global matching at test \\
    Topology & Connectivity/hole/contact proxies & Distance/skeleton approximations \\
    Calibration & Temp. scaling; ECE/ACE & Threshold selection; QA triage \\
    \hline
  \end{tabular}
  \caption{Loss components aligned with the three goals and priors.}
  \label{tab:dl-loss-compare}
\end{table}

\begin{table}[t]
  \centering
  \scriptsize
  \setlength{\tabcolsep}{2pt}\renewcommand{\arraystretch}{1.05}
  \begin{tabular}{@{} p{0.32\linewidth} p{0.66\linewidth} @{}}
    \hline
    \textbf{Category} & \textbf{Examples} \\
    \hline
    Intensity/noise & Brightness/contrast/gamma; histogram match; Poisson/Gaussian; missing-slice sim. \\
    Geometry & In-plane rotations \((0/90/180/270^\circ)\), flips, elastic warps, light scaling \\
    Semi-supervision & Teacher--student weak/strong consistency; conf.-based filtering; morphology checks \\
    Domain adaptation & Histogram/style transfer (lightweight); feature alignment (MMD/adversarial) \\
    \hline
  \end{tabular}
  \caption{Training-time robustness and label-efficiency strategies.}
  \label{tab:dl-aug-compare}
\end{table}

\subsection{Structural priors and constraints}

\noindent\textbf{Scope and taxonomy}\;
We categorize \emph{structural priors/constraints} into four complementary families and indicate where each is enforced:
(i) \emph{biological priors} (membrane contact, thin-sheet thickness, vesicle/mitochondria proximity, polyadic multiplicity);
(ii) \emph{geometric priors} (smoothness, perimeter/curvature control, thickness window, shape compactness);
(iii) \emph{topological priors} (connectivity, no spurious perforations, Euler-characteristics control);
(iv) \emph{graph-level priors} (pre--post edge legality, degree and distance statistics, cross-block consistency).
Realization layers include \underline{differentiable losses}, \underline{differentiable proxies}, \underline{decoding projections/post-processing}, and \underline{post-training regularization}. All distances, radii, areas, and thicknesses are reported in \emph{physical units} (nm or \(\mu\)m).\par

\medskip
\noindent\textbf{Notation and physical units}\;
Let \(\hat S\in[0,1]^{H\times W\times D}\) denote a synapse-probability volume; \(M\in[0,1]^{H\times W\times D}\) a membrane probability or mask; \(\mathrm{Dist}(\cdot)\) a distance transform defined in physical units.
We use \(\rho\) for contact radius, \([\tau_{\min},\tau_{\max}]\) for admissible thickness window, \(A_{\min}\) for minimal contact area, and \(r\) for matching radius (all in nm unless noted).\par

\medskip
\noindent\textbf{Biological context priors}\;
Synaptic interfaces arise at membranes with dense vesicles near presynaptic active zones juxtaposed to postsynaptic densities.
Models benefit from auxiliary heads (membrane affinity, vesicle likelihood, partner-offset vectors) and from rules that allow \emph{polyadic} one-to-many assignments in insects while discouraging artificial one-to-one couplings.
When neuron masks are available, pre/post assignments are constrained to distinct neurites with nonzero membrane contact and bounded centroid distance in physical units.\par

\medskip
\noindent\textbf{Geometric and topological regularization}\;
Thin interfaces are fragile under weak supervision.
Losses typically combine data fidelity with connectivity, hole, and boundary-consistency terms, and enforce hierarchy across vesicles \(\rightarrow\) active zones \(\rightarrow\) synapse masks.
We formalize the components below as differentiable (or differentiably approximated) objectives and provide projection operators for inference-time enforcement.\par

% --------------------------- Loss Set S (formulas) ---------------------------
\medskip
\noindent\textbf{Loss Set S: priors as differentiable (or proxy) objectives}\;

\noindent\textit{S1. Membrane-contact prior}\;
Let \(\mathrm{DT}_M=\mathrm{SignedDist}(M)\) be negative inside membranes.
\[
L_{\mathrm{mc}}
=\frac{1}{|\Omega|}\sum_{\mathbf r}\hat S(\mathbf r)\cdot \max\!\bigl(0,\,\mathrm{DT}_M(\mathbf r)-\rho\bigr),
\qquad
L_{\mathrm{area}}
=\max\!\bigl(0,\,A_{\min}-|\hat S\cap \mathrm{Dilate}(M,\rho)|\bigr).
\]

\noindent\textit{S2. Thin-sheet thickness window}\;
With a soft skeleton \(\mathrm{Skel}(\hat S)\) and thickness estimate \(\mathrm{Thick}(\hat S)\) (via distance transforms):
\[
L_{\mathrm{thick}}
=\frac{1}{|\Omega|}\sum_{\mathbf r}\hat S(\mathbf r)\Bigl[
\max\!\bigl(0,\,\mathrm{Thick}(\hat S)(\mathbf r)-\tau_{\max}\bigr)
+\max\!\bigl(0,\,\tau_{\min}-\mathrm{Thick}(\hat S)(\mathbf r)\bigr)\Bigr].
\]

\noindent\textit{S3. Geometric smoothness (TV/length/curvature)}\;
\[
L_{\mathrm{tv}}=\sum_{\mathbf r}\lVert\nabla \hat S(\mathbf r)\rVert_1,
\qquad
L_{\mathrm{len}}=\sum_{\partial \hat S}1,
\qquad
L_{\mathrm{curv}}\approx \sum_{\partial \hat S}\kappa^2
\ \ (\text{finite-difference approximation}).
\]

\noindent\textit{S4. Topology: connectivity and hole suppression}\;
A soft connectivity proxy (e.g., LogSumExp paths) with representative points \(\mathcal C\):
\[
L_{\mathrm{conn}}=1-\frac{1}{Z}\sum_{c\in\mathcal C}\mathrm{SoftConn}(\hat S;c).
\]
A small-cycle penalty as a proxy to persistent homology:
\[
L_{\mathrm{hole}}\approx \sum_{b\in \text{small-cycles}} w_b\,\sigma\!\bigl(\mathrm{Pers}_b-\epsilon\bigr).
\]

\noindent\textit{S5. Graph-level priors (edge legality and degrees)}\;
With predicted affinity \(\hat A_{ij}\), distance \(d_{ij}\), and membrane-contact gate \(\chi_{ij}\in\{0,1\}\):
\[
L_{\mathrm{edge}}
=\sum_{i,j}\hat A_{ij}\Bigl[\alpha\,\max(0,\,d_{ij}-d_{\max})+\beta\,(1-\chi_{ij})\Bigr],
\qquad
L_{\mathrm{deg}}=\mathrm{KL}\bigl(p_{\mathrm{deg}}^{\mathrm{pred}}\;\|\;p_{\mathrm{deg}}^{\mathrm{prior}}\bigr).
\]

\noindent\textit{S6. Multi-task consistency (membrane/boundary/skeleton)}\;
\[
L_{\mathrm{cons}}
=\lambda_m\,\mathrm{BCE}\!\bigl(\mathbb{1}[\hat S>0.5],\,\mathrm{Dilate}(M,\rho)\bigr)
+\lambda_b\,\mathrm{Dice}\bigl(\hat B,\,\partial \hat S\bigr)
+\lambda_k\,\lVert \mathrm{Skel}(\hat S)-\hat K\rVert_1,
\]
where \(\hat B\) and \(\hat K\) are boundary/skeleton heads.

\noindent\textit{S7. Structured objective (weighted sum)}\;
\begin{eqnarray}
L_{\text{struct}} &=& \lambda_{\mathrm{mc}}L_{\mathrm{mc}}+\lambda_{\mathrm{area}}L_{\mathrm{area}}
+\lambda_{\mathrm{thick}}L_{\mathrm{thick}}+\lambda_{\mathrm{tv}}L_{\mathrm{tv}} \nonumber \\
&&\quad + \lambda_{\mathrm{conn}}L_{\mathrm{conn}}+\lambda_{\mathrm{hole}}L_{\mathrm{hole}}
+\lambda_{\mathrm{edge}}L_{\mathrm{edge}}+\lambda_{\mathrm{deg}}L_{\mathrm{deg}}
+\lambda_{\mathrm{cons}}L_{\mathrm{cons}}.
\end{eqnarray}
Select \(\lambda_{\cdot}\) via grid search, uncertainty weighting, or GradNorm; report sensitivity curves.\par

% --------------------------- Projection / decoding ---------------------------
\medskip
\noindent\textbf{Projection operators (inference-time enforcement)}\;
\begin{enumerate}
  \item \textbf{Small-hole fill} (area \(<\alpha\)); \textbf{Largest-CC} per instance.
  \item \textbf{Thickness clamp} via morphology on a distance field to restrict thickness to \([\tau_{\min},\tau_{\max}]\).
  \item \textbf{Membrane snap}: project non-contact voxels of \(\hat S\) along \(\nabla \mathrm{DT}_M\) to the nearest membrane within \(\rho\); otherwise drop.
  \item \textbf{Edge projection}: cost \(C_{ij}=-\log \hat A_{ij}+\gamma\,\max(0,\,d_{ij}-d_{\max})\); solve a bipartite matching with capacity/degree bounds via Hungarian or min-cost flow.
\end{enumerate}

% --------------------------- Consistency and stability -----------------------
\medskip
\noindent\textbf{Consistency, stability, and gating}\;
Distance/skeleton operators may use precomputed or smoothed proxies; clip gradients for small-cycle terms; warm up with data/geometric losses before increasing topology/graph weights.
Down-weight \(L_{\mathrm{mc}}\) in low-membrane-confidence regions.
For semi-supervision, accept pseudo-labels only if structural violation rates are below a threshold, and increase \(L_{\text{struct}}\) on violations.\par

% --------------------------- Structural compliance metrics -------------------
\medskip
\noindent\textbf{Structural compliance metrics}\;
Beyond Dice/VOI/Rand \(F\), report:
(i) \emph{membrane-contact violation rate} (voxel-/instance-level);
(ii) \emph{thickness out-of-window rate} (mean and 95\% quantiles);
(iii) \emph{connectivity rate} (success of \(s\)--\(t\) connectivity or CC-count distribution);
(iv) \emph{hole rate / Euler error};
(v) \emph{graph legality} (illegal-edge rate, degree-distribution KL, distance-distribution shift).
Attach 95\% confidence intervals via bootstrap over instances/tiles.\par

% --------------------------- Mapping table (compact) -------------------------
\medskip
\begin{table}[t]
  \centering
  \scriptsize
  \setlength{\tabcolsep}{2pt}\renewcommand{\arraystretch}{1.05}
  \begin{tabular}{@{} p{0.20\linewidth} p{0.33\linewidth} p{0.29\linewidth} p{0.16\linewidth} @{}}
    \hline
    \textbf{Prior} & \textbf{Differentiable/approx.\ loss} & \textbf{Projection / postproc.} & \textbf{Notes (extra heads/compute)} \\
    \hline
    Membrane contact & \(L_{\mathrm{mc}}, L_{\mathrm{area}}\) & Snap-to-membrane or prune beyond \(\rho\) & Needs membrane head \(M\) or external mask \\
    Thickness window & \(L_{\mathrm{thick}}\) & Open/close on distance field, clamp to \([\tau_{\min},\tau_{\max}]\) & Distance/skeleton proxy \\
    Smoothness/length & \(L_{\mathrm{tv}}, L_{\mathrm{len}}, L_{\mathrm{curv}}\) & Boundary refinement & Synergy with boundary head \(\hat B\) \\
    Connectivity & \(L_{\mathrm{conn}}\) (soft paths/max-flow proxy) & Keep largest CC & Increase weight gradually \\
    Hole suppression & \(L_{\mathrm{hole}}\) (small cycles) & Fill small holes (\(<\alpha\)) & Enable only on small loops \\
    Graph legality & \(L_{\mathrm{edge}}, L_{\mathrm{deg}}\) & Hungarian/min-cost flow with capacities & Needs \(\hat A\) or embeddings \(\hat Z\) \\
    Multi-task consistency & \(L_{\mathrm{cons}}\) & -- & Needs boundary/skeleton heads \\
    \hline
  \end{tabular}
  \caption{Mapping structural priors to training losses and inference-time projections. All thresholds are in physical units.}
  \label{tab:priors-mapping}
\end{table}

% --------------------------- Failure modes and mitigation --------------------
\medskip
\noindent\textbf{Failure modes and mitigation}\;
Strong anisotropy or low contrast can shift membranes and trigger false contact violations; extremely thin clefts can be over-penalized by thickness terms; crowded polyadic regions may cause tension between connectivity and hole proxies.
Mitigate via confidence-gated \(L_{\mathrm{mc}}\), relaxed thickness windows with data-driven schedules, and local re-estimation around hard regions.\par

\subsection{Benchmarks, metrics \& failure modes}

A rigorous evaluation stack for synapse detection and segmentation should cover: (i) public datasets with fixed splits and annotation granularity; (ii) a metric suite aligned with detection, segmentation, topology, and graph objectives under \emph{unified matching rules}; (iii) engineering indicators (throughput, memory, proofreading cost); and (iv) a failure taxonomy with diagnostic visualizations and protocol-level remedies.

\medskip
\noindent\textbf{Datasets (compact summaries)}\;
Table~\ref{tab:benchmarks-core} and Table~\ref{tab:benchmarks-labels} summarize commonly used benchmarks with a \emph{consistent schema} (modality, physical sampling, volume, label granularity, density, polyadic ratio, license, recommended split). When adding 2024--2025 corpora, extend these tables using the same fields to ensure comparability.

% ---- BM1a: Core facts (kept narrow to fit a page) --------------------------
\medskip
\begin{table}[t]
  \centering
  \scriptsize
  \setlength{\tabcolsep}{2pt}\renewcommand{\arraystretch}{1.05}
  \begin{tabular}{@{} l l l l l @{}}
    \hline
    \textbf{Dataset} & \textbf{Modality / sampling} & \textbf{Voxel size (nm)} & \textbf{Volume (vox)} & \textbf{License \& split} \\
    \hline
    CREMI & ssEM (anisotropic) & \(\Delta_x\times\Delta_y\times\Delta_z\) & \(H\times W\times D\) & public; std.\ train/val/test \\
    FAFB / FlyWire & ssEM, whole brain & \(\Delta_x\times\Delta_y\times\Delta_z\) & full brain; region subsets & public access; region splits \\
    SNEMI3D & ssEM (anisotropic) & \(\Delta_x\times\Delta_y\times\Delta_z\) & single challenge block & challenge split \\
    Lab microvols & ssEM or FIB--SEM & \(\Delta_x\times\Delta_y\times\Delta_z\) & small blocks & per-lab license; ad hoc split \\
    \hline
  \end{tabular}
  \caption{Benchmarks (core). Fill voxel sizes, volumes, and formal licenses in the camera-ready version.}
  \label{tab:benchmarks-core}
\end{table}

% ---- BM1b: Labels and usage (kept narrow to fit a page) --------------------
\medskip
\begin{table}[t]
  \centering
  \scriptsize
  \setlength{\tabcolsep}{2pt}\renewcommand{\arraystretch}{1.05}
  \begin{tabular}{@{} l l l l l @{}}
    \hline
    \textbf{Dataset} & \textbf{Annotations} & \textbf{Syn.\ density (/\,\(\mu\mathrm{m}^3\))} & \textbf{Polyadic (\%)} & \textbf{Primary use} \\
    \hline
    CREMI & cleft masks; pre/post partners & value & value & site/cleft \& pairing validation \\
    FAFB / FlyWire & sites; partner assignments & value & value & large-scale detection \& graph analysis \\
    SNEMI3D & membranes; limited synapses & value & value & thin-structure stress tests \\
    Lab microvols & points/lines/sparse masks & value & value & cross-domain generalization \\
    \hline
  \end{tabular}
  \caption{Benchmarks (labels). Add densities/polyadic ratios and standard citations in the camera-ready version.}
  \label{tab:benchmarks-labels}
\end{table}

\medskip
\noindent\textbf{Box M1: matching \& evaluation protocol (defaults and ranges)}\;
\emph{Detection (points/sites).} One-to-one matching by \emph{physical} distance:
\[
d_{\mathrm{phys}}(p,g)=\big\lVert (\mathbf r_p-\mathbf r_g)\odot \Delta \big\rVert_2 \le r,
\]
processed in descending confidence; default \(r=\textit{XX}\,\mathrm{nm}\) with sensitivity sweeps.  
\emph{Instance masks.} Either (A) 1--1 matching by IoU\(\ge \tau_{\mathrm{IoU}}\) with \(\tau_{\mathrm{IoU}}=0.5\), or (B) center-distance matching with IoU as a quality attribute; \emph{fix one option} and use it consistently.  
\emph{Boundary tolerance.} Boundary-F1 uses a dilation band of radius \(\delta=\textit{YY}\,\mathrm{nm}\).  
\emph{Graph (pairing).} Treat pre--post pairs as edges (state directed/undirected); restrict candidates to a local radius \(r\); training may use soft-assignment, evaluation reports PR/ROC/AUC; edges violating \(d_{\max}\) or membrane contact are counted as negatives.  
\emph{Aggregation.} Report both micro-averages (pool instances) and macro-averages (per-dataset averages), and specify the primary score.

\medskip
\noindent\textbf{Metrics Set B: formulas and implementation notes}\;
\emph{Detection: PR/AP and \(F_1@r\).} With the 1--1 match set \(\mathcal M\),
\[
\mathrm{Precision}=\frac{|\mathcal M|}{N_{\mathrm{pred}}},\quad
\mathrm{Recall}=\frac{|\mathcal M|}{N_{\mathrm{gt}}},\quad
\mathrm{F1@}r=\frac{2\,\mathrm{Prec}\cdot\mathrm{Rec}}{\mathrm{Prec}+\mathrm{Rec}}.
\]
AP is the numerical area under the PR curve (state whether using full integration or 11-point interpolation).  
\emph{Segmentation: Dice / mIoU / Boundary-F1@\(\delta\).}
\[
\mathrm{Dice}=\frac{2|\hat S\cap S|}{|\hat S|+|S|},\qquad
\mathrm{IoU}=\frac{|\hat S\cap S|}{|\hat S\cup S|}.
\]
Let \(B(\cdot)\) be the contour set and \(\mathrm{Dil}_\delta(\cdot)\) a radius-\(\delta\) dilation:
\[
\mathrm{Prec}_b=\frac{|B(\hat S)\cap \mathrm{Dil}_\delta(B(S))|}{|B(\hat S)|},\ \ 
\mathrm{Rec}_b=\frac{|B(S)\cap \mathrm{Dil}_\delta(B(\hat S))|}{|B(S)|},\ \ 
\mathrm{BF1}=\frac{2\,\mathrm{Prec}_b\,\mathrm{Rec}_b}{\mathrm{Prec}_b+\mathrm{Rec}_b}.
\]
For thin interfaces, optionally add \(H_{95}\) (95th-percentile Hausdorff) or ASD.  
\emph{Instance partition: VOI / Rand \(F\).}
\[
\mathrm{VOI}=H(\hat{\mathcal Y}\mid\mathcal Y)+H(\mathcal Y\mid\hat{\mathcal Y}),
\]
with entropies estimated from voxel frequencies; use either Rand \(F\) or ARI (fix one across the thesis).  
\emph{Graph/pairing: edge PR/ROC/AUC and legality.}
\[
\mathrm{Edge\;Prec}=\frac{|\mathcal E^{\mathrm{pred}}\cap \mathcal E^{\mathrm{gt}}|}{|\mathcal E^{\mathrm{pred}}|},\qquad
\mathrm{Edge\;Rec}=\frac{|\mathcal E^{\mathrm{pred}}\cap \mathcal E^{\mathrm{gt}}|}{|\mathcal E^{\mathrm{gt}}|}.
\]
Additionally report illegal-edge rate (distance/contact violations) and degree-distribution divergence (e.g., KL).  
\emph{Calibration: ECE / Brier.} With \(B\) bins,
\[
\mathrm{ECE}=\sum_{b=1}^{B}\frac{n_b}{n}\bigl|\mathrm{acc}(b)-\mathrm{conf}(b)\bigr|,\qquad
\mathrm{Brier}=\frac{1}{n}\sum_{i=1}^n\bigl(\hat p_i-y_i\bigr)^2,
\]
and fix temperature \(T\) by validation scaling.

\medskip
\noindent\textbf{Statistical confidence \& significance}\;
Use stratified bootstrap over instances/tiles (e.g., 1{,}000 resamples) to report 95\% CIs.
For pairwise method comparisons, apply a paired randomization test or Wilcoxon signed-rank on per-tile scores; state the test and \(p\)-value handling in the caption or appendix.

\medskip
\noindent\textbf{Engineering indicators and QA cost}\;
Report (i) \emph{proofreading minutes per \(10^3\) synapses}, estimated from calibrated scores and uncertainty thresholds; (ii) \emph{inference throughput} (vox/s or \(\mu\mathrm{m}^3\)/s) with tile/halo settings; (iii) \emph{peak memory}, parameters/FLOPs, latency; (iv) hardware and library versions. When reporting whole-brain throughput, include block size, overlap, fusion rule, and parallelization scheme.

\medskip
\noindent\textbf{Cross-domain/generalization \& partial labels}\;
For cross-dataset evaluations, stratify by domain factors (region/species/instrument/batch) and keep \((r,\delta,\tau_{\mathrm{IoU}})\) identical across domains.
For incomplete labels, use a \emph{mask-of-interest} to exclude unlabeled regions from TP/FP/FN and \emph{report the number of evaluable instances}.

\medskip
\noindent\textbf{Failure taxonomy F and diagnostics}\;
\emph{Missed detections} (low contrast, faint clefts, dark cytosol) \(\Rightarrow\) mitigate with context aggregation, label/feature augmentation, hard-region mining.  
\emph{Over-/under-segmentation} (membrane adhesion, thin-layer breaks) \(\Rightarrow\) boundary-aware losses, connectivity regularizers, cross-tile instance reconciliation.  
\emph{Block-boundary artefacts} (misalignment, stitching, resampling) \(\Rightarrow\) deformable alignment, overlap inference with consensus merging.  
\emph{Cross-domain shift} (staining/species/region) \(\Rightarrow\) self-supervised pretraining, adaptation, calibrated triage.  
\emph{Partner misassignment} in polyadic neuropil \(\Rightarrow\) offset-vector heads \(+\) neurite-identity constraints; bipartite pruning by distance and membrane contact.  

\medskip
\noindent\textbf{Algorithm E1: standard evaluation pipeline}\;
\begin{enumerate}
  \item \textbf{Preprocess}:\; resample predictions and ground truth to the same \emph{physical} grid; crop to ROI and ignore unlabeled areas.
  \item \textbf{Detection}:\; greedy/Hungarian 1--1 matching under radius \(r\)\(\to\) TP/FP/FN \(\to\) PR/AP and \(F_1@r\).
  \item \textbf{Masks/boundary}:\; compute Dice/mIoU; build a \(\delta\)-band for Boundary-F1.
  \item \textbf{Instances/topology}:\; compute VOI and Rand \(F\) (or ARI); optionally structural compliance (contact/thickness/connectivity/hole rates).
  \item \textbf{Graph/pairing}:\; threshold local candidates into edges; compute edge PR/ROC/AUC, illegal-edge rate, degree KL.
  \item \textbf{Calibration}:\; temperature scaling on validation; fix thresholds; export reliability curves.
  \item \textbf{Confidence}:\; stratified bootstrap (1{,}000 resamples) for 95\% CIs; use paired tests for method deltas.
  \item \textbf{Runtime}:\; record throughput, memory, latency, hardware/libraries; log tile/halo/TTA/fusion settings.
\end{enumerate}

\subsection{Trends and open issues}

This subsection is organized as a four-link template for actionable research: \emph{trend axis} \(\rightarrow\) \emph{key challenge} \(\rightarrow\) \emph{testable hypothesis} \(\rightarrow\) \emph{minimal validation experiment / metrics / risks}. The goal is to pose \emph{falsifiable} and \emph{reproducible} questions, aligned with unified evaluation rules (matching radius \(r\), boundary tolerance \(\delta\), IoU threshold \(\tau_{\mathrm{IoU}}\)) reported in physical units (nm/\(\mu\)m) and consistent with the metrics subsection.

\medskip
\noindent\textbf{Trend A: end-to-end detect--segment--assign}\;
\emph{Key challenge:} reduce cascade error and threshold coupling while optimizing graph-level fidelity. 
\emph{Hypothesis:} training with a differentiable pairing surrogate and a graph-aware loss improves edge PR/AUC and calibration at fixed voxel/mask quality. 
\emph{Minimal experiment:} keep the backbone fixed; compare (i) two-stage cascade versus (ii) joint heads with soft assignment; evaluate at identical tiling/halo and operating points. 
\emph{Primary metrics:} \(F_1@r\) (sites), Dice/mIoU \(\&\) Boundary-F1@\(\delta\) (masks), edge PR/ROC/AUC \(\&\) ECE (graph), plus proofreading minutes per \(10^3\) synapses. 
\emph{Risks/ethics:} over-regularization may hide rare motifs; publish failure cases and calibration plots.

\medskip
\noindent\textbf{Trend B: cross-domain transfer and low-shot learning}\;
\emph{Key challenge:} generalize across species/regions/protocols under sparse labels. 
\emph{Hypothesis:} EM-specific self-supervised pretraining (masked/context) and lightweight feature alignment reduce domain-gap at constant label budget. 
\emph{Minimal experiment:} label-budget curve on target domain (1\%, 2\%, 5\%, 10\%); compare plain supervised vs.\ pretrain+adapt; stratify by domain factors. 
\emph{Primary metrics:} micro/macro \(F_1@r\), Boundary-F1@\(\delta\), VOI/Rand \(F\), edge AUC; report \(\Delta\) (out-of-domain minus in-domain). 
\emph{Risks/ethics:} negative transfer and batch effects; document domain statistics and access approvals.

\medskip
\noindent\textbf{Trend C: uncertainty, calibration, and QA cost}\;
\emph{Key challenge:} small-object imbalance yields miscalibration and volatile thresholds. 
\emph{Hypothesis:} temperature scaling and uncertainty-aware triage minimize expected QA time at fixed accuracy. 
\emph{Minimal experiment:} sweep temperatures \(T\) and thresholds; produce reliability curves and QA-cost frontiers. 
\emph{Primary metrics:} ECE/Brier, PR at fixed QA budget, edge illegal-rate; report 95\% CIs by bootstrap. 
\emph{Risks/ethics:} overly smooth probabilities may defer hard cases; disclose triage heuristics.

\medskip
\noindent\textbf{Trend D: reproducibility, openness, and sustainability}\;
\emph{Key challenge:} heterogeneous protocols impede fair comparison and replication. 
\emph{Hypothesis:} standardized splits, physical-unit tolerances, and containerized pipelines close reproducibility gaps; model distillation/quantization preserves accuracy within \(\leq 1\%\) while doubling throughput. 
\emph{Minimal experiment:} re-run baselines under unified \(r,\delta,\tau_{\mathrm{IoU}}\); report compute/energy and distilled variants. 
\emph{Primary metrics:} primary task scores \(+\) 95\% CIs, throughput (vox/s or \(\mu\mathrm{m}^3\)/s), memory, latency. 
\emph{Risks/ethics:} license compliance and transparent data governance.

\medskip
\noindent\textbf{T--Match: differentiable pairing surrogate (training) and discrete matching (inference)}\;
Let \(\mathcal{P}_{\mathrm{pre}},\mathcal{P}_{\mathrm{post}}\) be pre/post candidates and define a cost
\[
C_{ij} \;=\; \alpha\, d_{\mathrm{phys}}(i,j) \;+\; \beta\,\mathbb{1}\!\left[\neg\text{membrane-contact}\right] \;-\; \log \hat A_{ij},
\]
with \(\hat A_{ij}\) a learned affinity and \(d_{\mathrm{phys}}\) measured in nm. A soft assignment uses an entropically regularized normalization (e.g., Sinkhorn) at temperature \(\tau\):
\[
\mathrm{SoftAssign}_\tau(C)\;=\;\mathrm{Sinkhorn}\!\left(\exp\!\left(-\frac{C}{\tau}\right)\right), 
\qquad
L_{\mathrm{pair}}^{\mathrm{soft}}\;=\;\mathrm{CE}\!\left(\mathrm{SoftAssign}_\tau(C),\,Y\right).
\]
The total training loss couples voxel-, mask-, pair-, structure-, and calibration-terms:
\[
\begin{aligned}
L_{\mathrm{total}}
&=\lambda_{\mathrm{det}} L_{\mathrm{det}}
\;+\;\lambda_{\mathrm{seg}} L_{\mathrm{seg}}
\;+\;\lambda_{\mathrm{pair}} L_{\mathrm{pair}}^{\mathrm{soft}} \\
&\quad
\;+\;\lambda_{\mathrm{topo}} L_{\mathrm{struct}}
\;+\;\lambda_{\mathrm{cal}}\,\mathrm{ECE},
\end{aligned}
\]
with weights tuned on validation. At inference, solve a discrete bipartite matching (Hungarian or min-cost flow) using the \emph{same} cost \(C\); report the soft/hard discrepancy as a percentage.

\medskip
\begin{table}[t]
  \centering
  \scriptsize
  \setlength{\tabcolsep}{1.8pt}\renewcommand{\arraystretch}{1.05}
  \begin{tabular}{@{} p{2.9cm} p{5.2cm} p{4.6cm} p{3.0cm} p{2.8cm} @{}}
    \hline
    \textbf{Trend axis} & \textbf{Falsifiable hypothesis} & \textbf{Minimal experiment} & \textbf{Primary metrics/tests} & \textbf{Risks/ethics} \\
    \hline
    End-to-end pairing & Soft-assignment training improves edge AUC at fixed Dice/BF1@\(\delta\) & Swap cascade head for soft pairing; same backbone/tiling & Edge PR/ROC/AUC, ECE; paired test & Over-regularization of rare motifs \\
    EM pretraining & MAE/contrastive pretrain reduces labels by \(\geq\)X\% for same \(F_1@r\) & Budget curve (1--10\% labels) & AULB, \(F_1@r\), CI & Negative transfer across labs \\
    Structural priors & Adding \(L_{\mathrm{struct}}\) lowers violation rates without hurt to Dice & With/without \(L_{\mathrm{struct}}\) & BF1@\(\delta\), violation rates, Dice & Too-strong priors reduce recall \\
    Semi/active learning & Fixed QA budget \(\Rightarrow\) higher \(F_1@r\) via hard-region mining & Triage by uncertainty+violation & \(F_1@r\), QA min/\(10^3\) & Sampling bias \\
    Synthetic data & Mild domain-rand.\ \(\Rightarrow\) \(+\Delta\) on real set after few-shot & Sim\(\to\)real with few-shot tune & Dice, Edge AUC & Sim-real mismatch \\
    Robustness & Feature alignment \(>\) aug.\ only on OOD & Source\(\to\)multi-target & \(\Delta\) OOD--ID, CI & Added complexity \\
    Calibration & Temperature scaling reduces false edges at fixed recall & Grid over \(T,\tau\) & ECE, PR@QA budget & Over-smoothing \\
    Sustainability & Distill/quantize to \(2\times\) throughput, \(\leq\)1\% drop & KD/8-bit vs.\ full & \(F_1\), throughput, memory & Numeric stability \\
    \hline
  \end{tabular}
  \caption{Trends mapped to hypotheses, experiments, metrics, and risks. Fill concrete numbers, splits, and citations in the camera-ready version; keep tolerances in physical units.}
  \label{tab:trends-matrix}
\end{table}

\medskip
\begin{enumerate}
  \item \textbf{Ablation control:} change one factor only (head/loss/augment/calibration); keep seeds/tiling/halo fixed; report 95\% CIs and paired tests.
  \item \textbf{Sensitivity sweeps:} scan \(r,\delta,\tau_{\mathrm{IoU}}\), temperature \(T\), and \(\tau\) (soft assignment); plot performance--parameter curves.
  \item \textbf{Label-budget curve:} log-scale budget (1\%, 2\%, 5\%, 10\%, \dots); report area-under-label-budget (AULB).
  \item \textbf{In-/out-of-domain stratification:} same operating point; report ID/OOD scores and \(\Delta\); include domain statistics.
  \item \textbf{Structural compliance:} add membrane/thickness/connectivity/hole \emph{violation rates} to main results with 95\% CIs.
  \item \textbf{Engineering ledger:} throughput, memory, params/FLOPs, latency/energy (if available); record TTA/halo/fusion strategies.
\end{enumerate}